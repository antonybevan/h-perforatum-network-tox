\section*{Abstract}

Network-based metrics are widely used to identify associations between compounds and diseases, assuming that proximity within a protein--protein interaction network reflects functional relevance. However, these metrics are often reported as Z-scores, which we demonstrate are fundamentally sensitive to the number of targets a compound possesses. This dependency introduces a systematic bias where compounds with broad polypharmacology appear statistically significant due to null distribution tightening (the Law of Large Numbers) rather than physical network reachability. Here, we systematically evaluate this bias using the human liver interactome and a controlled comparison of two constituents from \textit{Hypericum perforatum}. We show that conventional proximity Z-scores yield unstable rankings that reverse depending on network construction parameters. While a compound with many targets may achieve a higher Z-score, it can remain physically more distant from the disease module than a compound with fewer, high-leverage targets. We resolve this by utilizing random walk--based influence propagation and introducing a size-normalized metric: perturbation efficiency. Our results show that influence-based rankings are stable across varied network thresholds and correctly identify high-leverage modulators that proximity metrics miss. This study provides a methodological template for identifying and correcting statistical artifacts in network medicine, enabling more reliable risk assessment in complex biological systems.

\vspace{0.5em}
\noindent\textbf{Keywords:} network propagation, proximity metrics, metric robustness, drug-induced liver injury, \\
polypharmacology, Z-score confounding, perturbation efficiency.
