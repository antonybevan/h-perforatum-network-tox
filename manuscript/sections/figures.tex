\clearpage
\section*{Figures}

\begin{figure}[ht]
\centering
\includegraphics[width=\textwidth]{../figures/main/fig1_lollipop.pdf}
\caption{\textbf{Network context: target count and physical proximity to DILI genes.}
(\textbf{A}) Target count in the liver-expressed largest connected component.
Quercetin: 62 targets; Hyperforin: 10 targets.
(\textbf{B}) Shortest-path proximity ($d_c$) to 82 DILI-associated genes.
Hyperforin is physically closer ($d_c = 1.30$) than Quercetin ($d_c = 1.68$).
Z-scores represent deviation from degree-matched null expectation ($n = 1{,}000$ permutations).
Quercetin: Z = $-$5.44 ($p < 0.001$); Hyperforin: Z = $-$3.86 ($p < 0.001$).
Negative Z-scores indicate closer-than-random proximity.
Network: STRING v12.0 (confidence $\geq$900), GTEx v8 (liver TPM $\geq$1).}
\label{fig:fig1}
\end{figure}

\clearpage
\begin{figure}[ht]
\centering
\includegraphics[width=0.85\textwidth]{../figures/main/fig2_dumbbell.pdf}
\caption{\textbf{Instability of proximity Z-scores.}
Dumbbell plot showing the dissociation between shortest-path proximity (left) and random walk influence (right) at STRING confidence $\geq$900.
At this threshold, Quercetin appears more "significant" in Z-score but is physically more distant ($1.68$ vs $1.30$) from DILI genes.
Hyperforin: proximity Z = $-$3.86, influence Z = +10.12 ($p < 0.001$).
Quercetin: proximity Z = $-$5.44, influence Z = +4.55 ($p < 0.001$).
Influence quantified by random walk with restart (RWR; $\alpha = 0.15$).
$n = 1{,}000$ degree-matched permutations per compound.}
\label{fig:fig2}
\end{figure}

\clearpage
\begin{figure}[ht]
\centering
\includegraphics[width=0.9\textwidth]{../figures/main/fig3_ewi_waterfall.pdf}
\caption{\textbf{Expression weighting refines influence propagation.}
Waterfall decomposition of Z-score changes under expression-weighted influence (EWI).
Initial Hyperforin advantage: $\Delta$Z = +5.57 (RWR).
Hyperforin change: $-$1.14 (attenuation of signal through liver-active hubs).
Quercetin change: +1.24 (gain from high-expression nodes like CFB).
Residual Hyperforin advantage: $\Delta$Z = +3.19.
Both compounds remain significant under EWI: Hyperforin Z = +8.98 ($p < 0.001$); Quercetin Z = +5.79 ($p < 0.001$).
Expression weighting from GTEx v8 liver tissue.}
\label{fig:fig3}
\end{figure}

\clearpage
\begin{figure}[ht]
\centering
\includegraphics[width=\textwidth]{../figures/main/fig4_ptni_phase.pdf}
\caption{\textbf{Average network influence quantifies efficiency disparity.}
Phase plot of total influence versus target count.
Horizontal lines represent efficiency tiers (Efficiency/average influence = constant).
Hyperforin occupies a higher efficiency region despite fewer targets.
Efficiency/average influence values: Hyperforin = 0.1138 (RWR), 0.1330 (EWI); Quercetin = 0.0322 (RWR), 0.0493 (EWI).
Efficiency difference: \textbf{3.7}$\times$ (based on bootstrap mean comparison).
The observed influence represents an effect-size normalization (total steady-state mass on DILI genes); no independent permutation test was performed.}
\label{fig:fig4}
\end{figure}

\clearpage
\begin{figure}[ht]
\centering
\includegraphics[width=0.8\textwidth]{../figures/main/fig5_bootstrap.pdf}
\caption{\textbf{Bootstrap sensitivity analysis excludes target-count confounding.}
Density distribution of RWR influence scores from 100 random 10-target samples drawn from Quercetin's 62-target pool.
Shaded region: 95\% confidence interval (0.0160--0.0542).
Vertical line: Hyperforin observed influence (0.1138).
Hyperforin exceeds the entire bootstrap distribution (3.7$\times$ fold vs.\ mean).
This confirms that Hyperforin's advantage is not attributable to favorable target count.
Bootstrap is a robustness control; it does not provide independent statistical evidence.}
\label{fig:fig5}
\end{figure}

\clearpage
\begin{figure}[ht]
\centering
\includegraphics[width=0.65\textwidth]{../figures/main/fig6_chemsim.pdf}
\caption{\textbf{Chemical similarity control excludes structural confounding.}
Maximum Tanimoto similarity to DILIrank reference drugs.
Reference set: 542 DILI-positive, 365 DILI-negative drugs.
Hyperforin: max = 0.15 (DILI+), 0.20 (DILI$-$).
Quercetin: max = 0.21 (DILI+), 0.22 (DILI$-$).
Dashed line: 0.4 threshold for structural analog detection \citep{Maggiora2014}.
Neither compound is a structural analog of known hepatotoxins.
This orthogonal analysis excludes chemical class as an explanation for the observed network signal.
Fingerprints: Morgan (ECFP4), radius 2, 2048 bits.}
\label{fig:fig6}
\end{figure}
