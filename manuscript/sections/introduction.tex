\section{Introduction}

% Paragraph 1: What proximity means, why it's attractive, why it's incomplete
Network-based drug safety assessment commonly utilizes proximity to disease-associated genes as a prioritization criterion, assuming that compounds with more targets or closer network positions pose greater risk \citep{Hopkins2008, Barabasi2011, Guney2016, Menche2015}.
This intuition is attractive due to its computational simplicity and interpretability.
However, proximity rankings are typically reported as Z-scores relative to degree-matched null models.
While Z-scores are intended to quantify statistical significance, they are fundamentally sensitive to target set size: as the number of targets increases, the variance of the null distribution decreases (the Law of Large Numbers), leading to inflated significance for compounds with broad polypharmacology.
Whether such "inferential significance" reflects true biological influence, or whether it introduces systematic bias into toxicological prioritization, remains a critical question.

% Paragraph 2: why H. perforatum is the minimal example and the metric stress test
\textit{Hypericum perforatum} (St.\ John's Wort) offers a sharp stress test for this question.
The extract contains two bioactive constituents with contrasting pharmacological profiles: Hyperforin (10 validated targets, potent PXR activator) and Quercetin (62 targets, broad polypharmacology) \citep{Nahrstedt1997, Moore2000, Boots2008}.
Conventional network analysis, relying on proximity Z-scores, would predict greater DILI-associated risk for Quercetin despite Hyperforin's well-established role in clinical hepatotoxicity through PXR-mediated enzyme induction.
This system therefore allows us to evaluate whether proximity-based prioritization is stable across network construction parameters or whether it is confounded by target-count artifacts.

% Paragraph 3: What we do differently, what principle we enforce
In this study, we systematically evaluate the robustness of proximity-based and influence-based metrics.
We demonstrate that proximity Z-scores yield unstable, threshold-dependent rankings that are driven by null-distribution tightening rather than physical distance.
To resolve this, we utilize random walk--based influence propagation, which integrates over all paths weighted by transition probability and captures signal amplification through regulatory hubs \citep{Kohler2008}.
We utilize probability-normalized influence mass to quantify perturbation efficiency, and use bootstrap resampling to exclude target-count confounding.
Our results demonstrate that influence-based propagation provides a more stable and mechanistically aligned framework for comparative network toxicology, particularly when comparing compounds with asymmetric target sets.
