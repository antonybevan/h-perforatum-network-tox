\clearpage

\begin{center}
{\Large\textbf{Supplementary Information}}\\[1em]
{\large Perturbation efficiency resolves target-count bias in network proximity metrics: A controlled audit}\\[1em]
Antony Bevan
\end{center}

\vspace{2em}

\section*{Supplementary Tables}

% Reset table counter and change format to S1, S2...
\setcounter{table}{0}
\renewcommand{\thetable}{S\arabic{table}}
\renewcommand{\theHtable}{Supplement.\thetable}

% Table S1: Hyperforin Targets with Sources
\begin{table}[ht]
\centering
\caption{\textbf{Hyperforin target genes and literature sources.}
All 14 raw targets with UniProt IDs, gene symbols, and primary literature sources.
Targets marked with * are present in the liver-expressed LCC (STRING $\geq$900, GTEx TPM $\geq$1).}
\label{tab:S1}
\vspace{0.5em}
\small
\begin{tabular}{llll}
\toprule
\textbf{UniProt} & \textbf{Gene} & \textbf{In LCC} & \textbf{Source} \\
\midrule
O75469 & NR1I2 (PXR) & Yes* & \citep{Moore2000, Watkins2001} \\
P08684 & CYP3A4 & Yes* & \citep{Moore2000} \\
P11712 & CYP2C9 & Yes* & \citep{Obach2000} \\
P20813 & CYP2B6 & Yes* & \citep{Komoroski2004} \\
P08183 & ABCB1 & Yes* & \citep{Hennessy2002} \\
Q9UNQ0 & ABCG2 & Yes* & \citep{Assefa2004} \\
O15440 & ABCC2 & Yes* & \citep{Wang2004} \\
P31749 & AKT1 & Yes* & \citep{Quiney2007} \\
P08253 & MMP2 & Yes* & \citep{Quiney2007} \\
P14780 & MMP9 & Yes* & \citep{Quiney2006} \\
Q9Y210 & TRPC6 & No & \citep{Leuner2007} \\
P15692 & VEGFA & No & \citep{Quiney2006} \\
Q13794 & PMAIP1 & No & \citep{Hostanska2003} \\
Q12879 & GRIN1 & No & \citep{Kumar2006} \\
\bottomrule
\end{tabular}
\end{table}

% Table S2: Quercetin Targets Summary
\begin{table}[ht]
\centering
\caption{\textbf{Quercetin target curation summary.}
Target counts at each processing stage.}
\label{tab:S2}
\vspace{0.5em}
\begin{tabular}{lr}
\toprule
\textbf{Stage} & \textbf{Count} \\
\midrule
Raw targets (ChEMBL v31, CHEMBL159) & 122 \\
Excluded: non-human (mouse, rat, bacterial, viral) & 10 \\
Excluded: no UniProt mapping & 25 \\
Processed targets & 87 \\
Excluded: not liver-expressed (TPM $<$ 1) & 20 \\
Excluded: not in STRING LCC & 5 \\
Final targets in LCC & 62 \\
\bottomrule
\end{tabular}
\end{table}

% Table S3: DILI Genes Summary
\begin{table}[ht]
\centering
\caption{\textbf{DILI gene set curation.}
Genes associated with drug-induced liver injury from DisGeNET (UMLS C0860207).}
\label{tab:S3}
\vspace{0.5em}
\begin{tabular}{lr}
\toprule
\textbf{Stage} & \textbf{Count} \\
\midrule
Raw DILI genes (DisGeNET) & 127 \\
In STRING $\geq$700 liver LCC & 84 \\
In STRING $\geq$900 liver LCC & 82 \\
Excluded: miRNAs (not in PPI network) & 21 \\
Excluded: cytokines (not in LCC) & 12 \\
Excluded: other & 12 \\
\bottomrule
\end{tabular}
\end{table}

% Table S4: Five Overlapping Targets
\begin{table}[ht]
\centering
\caption{\textbf{Genes targeted by both compounds.}
Five genes present in both Hyperforin and Quercetin target sets.}
\label{tab:S4}
\vspace{0.5em}
\begin{tabular}{lll}
\toprule
\textbf{Gene} & \textbf{Protein} & \textbf{Function} \\
\midrule
ABCG2 & BCRP & Efflux transporter \\
AKT1 & Protein kinase B & Cell survival signaling \\
CYP3A4 & Cytochrome P450 3A4 & Drug metabolism \\
MMP2 & Matrix metalloproteinase-2 & Extracellular matrix remodeling \\
MMP9 & Matrix metalloproteinase-9 & Extracellular matrix remodeling \\
\bottomrule
\end{tabular}
\end{table}

% Table S5: Direct DILI Gene Connectivity
\begin{table}[ht]
\centering
\caption{\textbf{Direct DILI gene connectivity.}
Hyperforin targets with first-order (distance = 1) connections to DILI genes in the STRING network ($\geq$900).
DILI neighbors are genes present in the 82-gene DILI set.}
\label{tab:S5}
\vspace{0.5em}
\resizebox{\textwidth}{!}{%
\begin{tabular}{llcl}
\toprule
\textbf{Target} & \textbf{DILI Neighbors} & \textbf{N} & \textbf{Function} \\
\midrule
CYP3A4 & NR1I2, CYP2E1, UGT1A9, GSTM1, GSTP1 & 5 & Xenobiotic metabolism \\
AKT1 & MAP3K5, NFE2L2, CTNNB1, IGF1 & 4 & Stress response \\
MMP9 & LCN2, SPP1, MMP2 & 3 & Inflammation/ECM \\
ABCB1 & ABCC2, NR1I2 & 2 & Drug transport \\
CYP2C9 & CYP2E1, NR1I2 & 2 & Xenobiotic metabolism \\
CYP2B6 & NR1I2 & 1 & Xenobiotic metabolism \\
NR1I2 & CYP2E1, ABCC2 & 2 & Master regulator \\
ABCG2 & ABCC2 & 1 & Drug transport \\
ABCC2 & NR1I2, ABCB1 & 2 & Drug transport \\
MMP2 & MMP9, SPP1 & 2 & ECM remodeling \\
\midrule
\textbf{Total unique} & & \textbf{12} & \\
\bottomrule
\end{tabular}}
\end{table}

% Table S6: Quercetin Direct DILI Connectivity
\begin{table}[ht]
\centering
\caption{\textbf{Quercetin direct DILI gene connectivity summary.}
Summary statistics for first-order DILI connections across Quercetin's 62 targets.}
\label{tab:S6}
\vspace{0.5em}
\begin{tabular}{lr}
\toprule
\textbf{Metric} & \textbf{Value} \\
\midrule
Total targets in LCC & 62 \\
Targets with $\geq$1 direct DILI neighbor & 18 \\
Total direct DILI connections & 31 \\
Mean DILI neighbors per target & 0.50 \\
\addlinespace
\textit{Hyperforin comparison:} & \\
Hyperforin targets with $\geq$1 DILI neighbor & 10/10 (100\%) \\
Mean DILI neighbors per Hyperforin target & 2.4 \\
\bottomrule
\end{tabular}
\end{table}

% Table S7: Full Quercetin Target List
\begin{table}[ht]
\centering
\caption{\textbf{Quercetin target genes in the liver-expressed network.}
All 62 Quercetin targets in STRING v12.0 LCC (confidence $\geq$900) with liver TPM $\geq$1 (GTEx v8).
Sorted by descending liver expression.}
\label{tab:S7}
\vspace{0.5em}
\scriptsize
\begin{tabular}{lrlrlrlr}
\toprule
\textbf{Gene} & \textbf{TPM} & \textbf{Gene} & \textbf{TPM} & \textbf{Gene} & \textbf{TPM} & \textbf{Gene} & \textbf{TPM} \\
\midrule
CFB & 1115 & CYP3A4 & 335 & FN1 & 229 & ALDH2 & 183 \\
ANPEP & 160 & PPIA & 112 & SERPINA5 & 104 & CYP1A2 & 72 \\
CA2 & 64 & APP & 63 & PYGL & 55 & HDAC6 & 45 \\
ESRRA & 42 & MAOA & 35 & AKR1C2 & 33 & AKT1 & 33 \\
CTSH & 28 & XDH & 26 & CHRNA4 & 25 & PIK3R1 & 24 \\
PIM1 & 24 & LDLR & 23 & EGFR & 17 & ELOVL1 & 18 \\
PKN1 & 16 & GSK3A & 13 & YES1 & 13 & MET & 12 \\
DAPK1 & 12 & BACE1 & 11 & CSNK2A1 & 10 & FSTL1 & 9 \\
SIRT6 & 8 & GSK3B & 7 & CDK7 & 7 & CAV2 & 7 \\
PTPN2 & 6 & CYP1A1 & 5 & PRMT7 & 5 & MMP2 & 5 \\
AKR1B1 & 5 & PDE6D & 5 & PTK2 & 4 & ABCG2 & 4 \\
IQGAP1 & 4 & ADRB2 & 3 & BRAF & 4 & KDR & 3 \\
SRC & 3 & ALOX5 & 3 & CYP1B1 & 3 & TLR4 & 3 \\
NUAK1 & 3 & AXL & 2 & ADA & 2 & LCK & 2 \\
ABCC1 & 2 & PLK1 & 1 & ACHE & 1 & MMP9 & 1 \\
SYK & 1 & PDZK1IP1 & 1 & & & & \\
\bottomrule
\end{tabular}
\end{table}

% Table S8: DILI Gene List
\begin{table}[ht]
\centering
\caption{\textbf{DILI gene set (82 genes).}
Genes in STRING v12.0 LCC (confidence $\geq$900) with liver TPM $\geq$1 (GTEx v8).
Source: DisGeNET (UMLS C0860207). Sorted alphabetically.}
\label{tab:S8}
\vspace{0.5em}
\scriptsize
\begin{tabular}{llllllll}
\toprule
\multicolumn{8}{c}{\textbf{82 DILI-Associated Genes}} \\
\midrule
ABCB1 & AHR & ALB & ALDOB & AMBP & APOA1 & APOE & APOH \\
ARG1 & ARNT & ATG5 & BAX & BTD & C3 & CAT & CCL2 \\
CLU & COL3A1 & CTNNB1 & CXCL1 & CXCL10 & CYP2A6 & CYP2C19 & CYP2C9 \\
CYP2E1 & DGAT2 & ENO1 & FGA & FLT1 & FMO3 & GADD45A & GC \\
GCLC & GPT & GSN & GSTM1 & GSTM2 & GSTP1 & HLA-A & HLA-B \\
HLA-DQB1 & HLA-DRB1 & HMGB1 & HMOX1 & HPD & HPX & IGF1 & IL18 \\
IL1R2 & KRT18 & LCN2 & LGALS3 & MAP3K5 & MED1 & MMP2 & MTHFR \\
NAT2 & NFE2L2 & NR1H3 & NR1H4 & NR1I2 & NR1I3 & PLAT & PLG \\
PNP & POLG & PON1 & PPARA & PRKDC & PTGS2 & RBP1 & SLPI \\
SNX18 & SOD1 & SOD3 & SPP1 & TALDO1 & TBXA2R & TCTN1 & TF \\
TTR & UGT1A9 & & & & & & \\
\bottomrule
\end{tabular}
\end{table}

% Table S9: Permutation Test Summary
\begin{table}[ht]
\centering
\caption{\textbf{Null distribution parameters from permutation testing.}
Null distribution parameters (mean and standard deviation) from $n = 1{,}000$ degree-matched permutations.
Note the tightening of the Quercetin null distribution as the number of targets increases, which drives the inflation of proximity Z-scores.}
\label{tab:S9}
\vspace{0.5em}
\begin{tabular}{llrrrr}
\toprule
\textbf{Metric} & \textbf{Compound} & \textbf{$\mu_{null}$} & \textbf{$\sigma_{null}$} & \textbf{$x_{obs}$} & \textbf{Z-score} \\
\midrule
\multicolumn{6}{l}{\textit{Shortest-path proximity (at $\geq$900)}} \\
\addlinespace
& Hyperforin (10) & 2.21 & 0.235 & 1.30 & $-$3.86 \\
& Quercetin (62) & 2.17 & 0.091 & 1.68 & $-$5.44 \\
\addlinespace
\multicolumn{6}{l}{\textit{Random walk influence (at $\geq$900)}} \\
\addlinespace
& Hyperforin (10) & 0.0147 & 0.0098 & 0.1138 & +10.12 \\
& Quercetin (62) & 0.0148 & 0.0038 & 0.0322 & +4.55 \\
\addlinespace
\multicolumn{6}{l}{\textit{Expression-weighted influence (at $\geq$900)}} \\
\addlinespace
& Hyperforin (10) & 0.0205 & 0.0125 & 0.1330 & +8.98$^*$ \\
& Quercetin (62) & 0.0209 & 0.0049 & 0.0493 & +5.79$^*$ \\
\bottomrule
\end{tabular}

\vspace{0.3em}
\begin{minipage}{\linewidth}
\footnotesize
$^*$Significance remains high despite tissue-specific attenuation. $\mu_{null}$ = null distribution mean; $\sigma_{null}$ = null distribution standard deviation; $x_{obs}$ = observed metric value.
\end{minipage}
\end{table}

% Table S10: Robustness Control - Bootstrap (Moved from main)
\begin{table}[ht]
\centering
\caption{\textbf{Bootstrap sensitivity excludes target-count confounding.}
Random 10-target subsets ($n = 100$) sampled without replacement from Quercetin's 62-target pool.
Hyperforin's observed influence exceeds the entire bootstrap distribution.}
\label{tab:bootstrap}
\vspace{0.5em}
\begin{tabular}{@{}lcl@{}}
\toprule
\textbf{Statistic} & \textbf{Value} & \textbf{Interpretation} \\
\midrule
Hyperforin observed & 0.1138 & Reference \\
\addlinespace
Bootstrap mean & 0.0308 & Expected if targets equivalent \\
Bootstrap SD & 0.0100 & Sampling variability \\
Bootstrap 95\% CI & [0.0160, 0.0542] & 2.5th--97.5th percentile \\
\addlinespace
Hyperforin / mean & \textbf{3.7}$\times$ & Effect size \\
Exceeds 95\% CI? & \textbf{Yes} & Not attributable to sampling \\
\bottomrule
\end{tabular}

\vspace{0.3em}
\begin{minipage}{\linewidth}
\footnotesize
Random seed: 42. Note: Bootstrap confirms robustness to target selection; it does not constitute independent inferential evidence.
\end{minipage}
\end{table}

% Table S11: Exclusion Control - Chemical Similarity (Moved from main)
\begin{table}[ht]
\centering
\caption{\textbf{Chemical similarity excludes structural confounding.}
Neither compound resembles known hepatotoxins (Tanimoto $<$ 0.4).
Quercetin is more similar to DILI-positive drugs yet shows lower network influence.}
\label{tab:chemsim}
\vspace{0.5em}
\begin{tabular}{@{}lcccc@{}}
\toprule
\textbf{Compound} & \textbf{Max Tanimoto (DILI+)} & \textbf{Max Tanimoto (DILI$-$)} & \textbf{Analog?}$^*$ & \textbf{Network rank} \\
\midrule
Hyperforin & 0.154 & 0.202 & No & 1 (higher influence) \\
Quercetin & 0.212 & 0.220 & No & 2 (lower influence) \\
\bottomrule
\end{tabular}

\vspace{0.3em}
\begin{minipage}{\linewidth}
\footnotesize
$^*$Analog threshold: Tanimoto $>$ 0.4 (Maggiora et al., 2014). Morgan fingerprints (ECFP4, radius 2, 2048 bits). DILIrank: 542 DILI+, 365 DILI$-$ drugs.
\end{minipage}
\end{table}

% Table S12: Mechanistic Support - Target Profile (Moved from main)
\begin{table}[ht]
\centering
\caption{\textbf{Hyperforin targets include regulatory hubs.}
All 10 Hyperforin targets in the liver-expressed LCC, with liver expression (GTEx v8) and network degree.
PXR (NR1I2) is the master regulator; CYP enzymes are downstream effectors.}
\label{tab:hyperforin_targets}
\vspace{0.5em}
\begin{tabular}{@{}llrrll@{}}
\toprule
\textbf{Gene} & \textbf{Protein} & \textbf{TPM} & \textbf{Degree} & \textbf{Function} & \textbf{DILI link} \\
\midrule
NR1I2 & PXR & 43 & 28 & Master regulator & Direct \\
CYP3A4 & CYP3A4 & 335 & 89 & Xenobiotic metabolism & Direct \\
CYP2C9 & CYP2C9 & 434 & 76 & Xenobiotic metabolism & Direct \\
CYP2B6 & CYP2B6 & 125 & 42 & Xenobiotic metabolism & Indirect \\
AKT1 & PKB & 33 & \textbf{312} & Stress signaling & Indirect \\
ABCB1 & P-gp & 7 & 53 & Drug efflux & Direct \\
ABCC2 & MRP2 & 60 & 38 & Drug efflux & Direct \\
ABCG2 & BCRP & 4 & 31 & Drug efflux & Indirect \\
MMP2 & MMP2 & 5 & 87 & ECM remodeling & Indirect \\
MMP9 & MMP9 & 1 & 94 & ECM remodeling & Indirect \\
\bottomrule
\end{tabular}

\vspace{0.3em}
\begin{minipage}{\linewidth}
\footnotesize
AKT1 is the highest-degree target (312 neighbors). Five of 10 targets (NR1I2, CYP3A4, CYP2C9, ABCB1, ABCC2) are directly connected to DILI genes. TPM = transcripts per million; DILI = drug-induced liver injury; LCC = largest connected component.
\end{minipage}
\end{table}

% Table S13: Stability - Threshold Robustness (Moved from main)
\begin{table}[ht]
\centering
\caption{\textbf{Influence ranking is robust to network construction parameters.}
Hyperforin ranks first across all thresholds and influence metrics. 
Proximity Z-scores are unstable and reverse rankings between thresholds, failing to accurately reflect the physical distance advantage of Hyperforin.}
\label{tab:threshold_robustness}
\vspace{0.5em}
\begin{tabular}{@{}llrrrr@{}}
\toprule
\textbf{Threshold} & \textbf{Compound} & \textbf{RWR Z} & \textbf{EWI Z} & \textbf{Proximity $d_c$} & \textbf{Proximity Z} \\
\midrule
$\geq$700 & Hyperforin & \textbf{+12.08} & +11.20 & 0.60 & $-$6.04 \\
(11,693 nodes) & Quercetin & +5.53 & +7.09 & 1.34 & $-$5.46 \\
\addlinespace
$\geq$900 & Hyperforin & \textbf{+10.12} & +8.98 & 1.30 & $-$3.86 \\
(7,677 nodes) & Quercetin & +4.55 & +5.79 & 1.68 & $-$5.44 \\
\bottomrule
\end{tabular}

\vspace{0.3em}
\begin{minipage}{\linewidth}
\footnotesize
Note: At $\geq$900, Quercetin achieves a more "significant" proximity Z-score despite being physically more distant ($1.68$ vs $1.30$) from DILI genes. RWR = random walk with restart; EWI = expression-weighted influence; $d_c$ = mean minimum shortest-path distance; DILI = drug-induced liver injury.
\end{minipage}
\end{table}
