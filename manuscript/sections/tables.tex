\clearpage
\section*{Tables}

% ============================================================================
% Table 1: Primary Comparison - The Central Finding
% ============================================================================
\begin{table}[ht]
\centering
\caption{\textbf{Network metrics reveal the instability of proximity Z-scores.}
While Quercetin achieves more significant proximity Z-scores due to tighter null distributions, Hyperforin is physically closer ($d_c$) to DILI genes. Influence-based metrics resolve this confounding and stably prioritize Hyperforin.
Network: STRING v12.0 LCC (confidence $\geq$900) filtered to liver-expressed genes.}
\label{tab:main_results}
\vspace{0.5em}
\small
\begin{tabular}{@{}llccccc@{}}
\toprule
\textbf{Metric} & \textbf{Compound} & \textbf{Targets} & \textbf{Observed} & \textbf{Z} & \textbf{$p$} & \textbf{PTNI} \\
\midrule
\multicolumn{7}{@{}l}{\textit{Tier 1: Shortest-path proximity}} \\[0.3em]
& Hyperforin & 10 & $d_c = 1.30$ & $-$3.86 & $<0.001$$^*$ & --- \\
& Quercetin & 62 & $d_c = 1.68$ & \textbf{$-$5.44} & $<0.001$$^*$ & --- \\
& \multicolumn{5}{r}{\textit{Instability: Quercetin is physically more distant yet more "significant"}} \\[0.5em]
\multicolumn{7}{@{}l}{\textit{Tier 2: Random walk influence (RWR)}} \\[0.3em]
& Hyperforin & 10 & 0.1138 & \textbf{+10.12} & $<0.001$$^*$ & 0.1138 \\
& Quercetin & 62 & 0.0322 & +4.55 & $<0.001$ & 0.0322 \\
& \multicolumn{5}{r}{\textit{Resolution: Correctly prioritizes physical proximity and regulatory hub modulation}} \\[0.5em]
\multicolumn{7}{@{}l}{\textit{Tier 3: Expression-weighted influence (EWI)}} \\[0.3em]
& Hyperforin & 10 & 0.1330 & \textbf{+8.98} & $<0.001$$^*$ & 0.1330 \\
& Quercetin & 62 & 0.0493 & +5.79 & $<0.001$ & 0.0493 \\
\bottomrule
\end{tabular}

\vspace{0.3em}
\begin{minipage}{\linewidth}
\footnotesize
$^*$At permutation floor ($<$1/1,000).\\
PTNI = per-target network influence; RWR = random walk with restart; EWI = expression-weighted influence; $d_c$ = mean minimum shortest-path distance; DILI = drug-induced liver injury. All associations survived Benjamini--Hochberg FDR correction ($q < 0.05$).
\end{minipage}
\end{table}

% ============================================================================
% Table 2: Effect Size - The 3.7x Efficiency
% ============================================================================
\begin{table}[ht]
\centering
\caption{\textbf{Per-target influence efficiency.}
PTNI quantifies average influence per target, reframing polypharmacology as an efficiency problem.
Hyperforin targets are ~3.7-fold more efficient at perturbing the DILI module than Quercetin targets.}
\label{tab:ptni}
\vspace{0.5em}
\begin{tabular}{@{}lcccc@{}}
\toprule
\textbf{Analysis} & \textbf{Hyperforin PTNI} & \textbf{Quercetin PTNI} & \textbf{Efficiency Ratio}$^*$ & \textbf{Robust Ratio}$^\dagger$ \\
\midrule
RWR (topology-only) & 0.1138 & 0.0322 & \textbf{3.5}$\times$ & \textbf{3.7}$\times$ \\
EWI (expression-weighted) & 0.1330 & 0.0493 & \textbf{2.7}$\times$ & --- \\
\bottomrule
\end{tabular}

\vspace{0.3em}
\begin{minipage}{\linewidth}
\footnotesize
$^*$Efficiency Ratio = Observed PTNI ratio. $^\dagger$Robust Ratio = Observed influence / size-matched Bootstrap Mean (N=10). PTNI = per-target network influence; RWR = random walk with restart; EWI = expression-weighted influence.
\end{minipage}
\end{table}

% ============================================================================
% Table 3: Robustness Control - Bootstrap
% ============================================================================
\begin{table}[ht]
\centering
\caption{\textbf{Bootstrap sensitivity excludes target-count confounding.}
Random 10-target subsets ($n = 100$) sampled without replacement from Quercetin's 62-target pool.
Hyperforin's observed influence exceeds the entire bootstrap distribution.}
\label{tab:bootstrap}
\vspace{0.5em}
\begin{tabular}{@{}lcl@{}}
\toprule
\textbf{Statistic} & \textbf{Value} & \textbf{Interpretation} \\
\midrule
Hyperforin observed & 0.1138 & Reference \\
\addlinespace
Bootstrap mean & 0.0308 & Expected if targets equivalent \\
Bootstrap SD & 0.0100 & Sampling variability \\
Bootstrap 95\% CI & [0.0160, 0.0542] & 2.5th--97.5th percentile \\
\addlinespace
Hyperforin / mean & \textbf{3.7}$\times$ & Effect size \\
Exceeds 95\% CI? & \textbf{Yes} & Not attributable to sampling \\
\bottomrule
\end{tabular}

\vspace{0.3em}
\begin{minipage}{\linewidth}
\footnotesize
Random seed: 42. Note: Bootstrap confirms robustness to target selection; it does not constitute independent inferential evidence.
\end{minipage}
\end{table}

% ============================================================================
% Table 4: Exclusion Control - Chemical Similarity
% ============================================================================
\begin{table}[ht]
\centering
\caption{\textbf{Chemical similarity excludes structural confounding.}
Neither compound resembles known hepatotoxins (Tanimoto $<$ 0.4).
Quercetin is more similar to DILI-positive drugs yet shows lower network influence.}
\label{tab:chemsim}
\vspace{0.5em}
\begin{tabular}{@{}lcccc@{}}
\toprule
\textbf{Compound} & \textbf{Max Tanimoto (DILI+)} & \textbf{Max Tanimoto (DILI$-$)} & \textbf{Analog?}$^*$ & \textbf{Network rank} \\
\midrule
Hyperforin & 0.154 & 0.202 & No & 1 (higher influence) \\
Quercetin & 0.212 & 0.220 & No & 2 (lower influence) \\
\bottomrule
\end{tabular}

\vspace{0.3em}
\begin{minipage}{\linewidth}
\footnotesize
$^*$Analog threshold: Tanimoto $>$ 0.4 (Maggiora et al., 2014). Morgan fingerprints (ECFP4, radius 2, 2048 bits). DILIrank: 542 DILI+, 365 DILI$-$ drugs.
\end{minipage}
\end{table}

% ============================================================================
% Table 5: Mechanistic Support - Target Profile
% ============================================================================
\begin{table}[ht]
\centering
\caption{\textbf{Hyperforin targets include regulatory hubs.}
All 10 Hyperforin targets in the liver-expressed LCC, with liver expression (GTEx v8) and network degree.
PXR (NR1I2) is the master regulator; CYP enzymes are downstream effectors.}
\label{tab:hyperforin_targets}
\vspace{0.5em}
\begin{tabular}{@{}llrrll@{}}
\toprule
\textbf{Gene} & \textbf{Protein} & \textbf{TPM} & \textbf{Degree} & \textbf{Function} & \textbf{DILI link} \\
\midrule
NR1I2 & PXR & 43 & 28 & Master regulator & Direct \\
CYP3A4 & CYP3A4 & 335 & 89 & Xenobiotic metabolism & Direct \\
CYP2C9 & CYP2C9 & 434 & 76 & Xenobiotic metabolism & Direct \\
CYP2B6 & CYP2B6 & 125 & 42 & Xenobiotic metabolism & Indirect \\
AKT1 & PKB & 33 & \textbf{312} & Stress signaling & Indirect \\
ABCB1 & P-gp & 7 & 53 & Drug efflux & Direct \\
ABCC2 & MRP2 & 60 & 38 & Drug efflux & Direct \\
ABCG2 & BCRP & 4 & 31 & Drug efflux & Indirect \\
MMP2 & MMP2 & 5 & 87 & ECM remodeling & Indirect \\
MMP9 & MMP9 & 1 & 94 & ECM remodeling & Indirect \\
\bottomrule
\end{tabular}

\vspace{0.3em}
\begin{minipage}{\linewidth}
\footnotesize
AKT1 is the highest-degree target (312 neighbors). Five of 10 targets (NR1I2, CYP3A4, CYP2C9, ABCB1, ABCC2) are directly connected to DILI genes. TPM = transcripts per million; DILI = drug-induced liver injury; LCC = largest connected component.
\end{minipage}
\end{table}

% ============================================================================
% Table 6: Stability - Threshold Robustness
% ============================================================================
\begin{table}[ht]
\centering
\caption{\textbf{Influence ranking is robust to network construction parameters.}
Hyperforin ranks first across all thresholds and influence metrics. 
Proximity Z-scores are unstable and reverse rankings between thresholds, failing to accurately reflect the physical distance advantage of Hyperforin.}
\label{tab:threshold_robustness}
\vspace{0.5em}
\begin{tabular}{@{}llrrrr@{}}
\toprule
\textbf{Threshold} & \textbf{Compound} & \textbf{RWR Z} & \textbf{EWI Z} & \textbf{Proximity $d_c$} & \textbf{Proximity Z} \\
\midrule
$\geq$700 & Hyperforin & \textbf{+12.08} & +11.20 & 0.60 & $-$6.04 \\
(11,693 nodes) & Quercetin & +5.53 & +7.09 & 1.34 & $-$5.46 \\
\addlinespace
$\geq$900 & Hyperforin & \textbf{+10.12} & +8.98 & 1.30 & $-$3.86 \\
(7,677 nodes) & Quercetin & +4.55 & +5.79 & 1.68 & $-$5.44 \\
\bottomrule
\end{tabular}

\vspace{0.3em}
\begin{minipage}{\linewidth}
\footnotesize
Note: At $\geq$900, Quercetin achieves a more "significant" proximity Z-score despite being physically more distant ($1.68$ vs $1.30$) from DILI genes. RWR = random walk with restart; EWI = expression-weighted influence; $d_c$ = mean minimum shortest-path distance; DILI = drug-induced liver injury.
\end{minipage}
\end{table}
