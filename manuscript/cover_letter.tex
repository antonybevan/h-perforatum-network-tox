% ============================================================================
% Cover Letter for Computational Toxicology - Revised Professional Version
% ============================================================================
\documentclass[12pt]{article}
\usepackage[margin=1in]{geometry}
\usepackage{charter} % Professional font
\usepackage{hyperref}
\usepackage{setspace}

\pagestyle{empty}

\begin{document}

% --- Header / Contact Info ---
\begin{flushright}
    \textbf{Antony Bevan} \\
    Department of Pharmacognosy \\
    Madurai Medical College \\
    Madurai 625020, India \\
    Email: \url{antonybevan04@gmail.com} \\
    \medskip
    \today
\end{flushright}

\bigskip

% --- Recipient ---
\noindent
To the Editor-in-Chief \\
\textit{Computational Toxicology} \\
Elsevier Editorial Office

\bigskip

\noindent
\textbf{Subject: Submission of Original Research Article}

\medskip

\noindent
Dear Editor,

\onehalfspacing

I am pleased to submit our manuscript, ``\textbf{Comparative analysis of network-based measures for the assessment of drug-induced liver injury: A case study of \textit{Hypericum perforatum}},'' for consideration as an Original Research Article in \textit{Computational Toxicology}.

In this study, we address a critical methodological vulnerability in network-based toxicology: the sensitivity of proximity Z-scores to target-set size. Using \textit{H. perforatum} as a model, we demonstrate that traditional proximity-based rankings are fundamentally confounded by the law of large numbers, which leads to unstable, threshold-dependent significance assessments when comparing compounds with asymmetric target profiles.

To resolve this, we provide a robust framework based on random walk influence propagation and quantify \textbf{perturbation efficiency} as a normalized metric for comparative assessment. Our results show that influence-based metrics remain stable across network construction parameters and correctly identify the high-leverage modulation of the PXR axis by hyperforin, where proximity measures fail. 

We believe this work is particularly relevant to the readers of \textit{Computational Toxicology} as it identifies a systematic artifact in current prioritization workflows and provides a validated, reproducible alternative for comparative risk assessment. 

This manuscript is original and is not under consideration elsewhere. All code and curated data have been made available via a public repository (\url{https://github.com/antonybevan/h-perforatum-network-tox}) to ensure complete transparency and reproducibility.

Thank you for your time and for considering our work.

\bigskip

\noindent
Sincerely,

\medskip

\noindent
Antony Bevan

\end{document}
