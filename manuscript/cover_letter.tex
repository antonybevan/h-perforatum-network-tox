% Cover Letter for Scientific Reports
\documentclass[12pt]{article}
\usepackage[margin=1in]{geometry}
\usepackage{charter}
\usepackage{hyperref}
\usepackage{setspace}

\pagestyle{empty}

\begin{document}



\bigskip

\noindent
Editor-in-Chief \\
\textit{Scientific Reports}

\bigskip

\noindent
Dear Editor,

\onehalfspacing

We request consideration of the enclosed manuscript, ``Systematic bias in network proximity Z-scores: A comparative robustness audit using \textit{Hypericum perforatum} constituents,'' for publication in \textit{Scientific Reports}.

The manuscript compares proximity-based and influence-based network metrics for toxicity prediction. Using Hyperforin and Quercetin as test compounds, we show that proximity Z-scores can produce misleading rankings when target set sizes differ substantially. Specifically, Quercetin (62 targets) ranks as more significant than Hyperforin (10 targets) by proximity, despite Hyperforin being closer to DILI genes and being the known hepatotoxic agent in this system.

The effect arises from the law of large numbers: larger target sets yield tighter null distributions, which inflate Z-scores independently of actual topological proximity. Random walk influence metrics do not exhibit this behavior and rank Hyperforin consistently across thresholds.

Bootstrap resampling and chemical similarity controls are included to address potential confounders. All code and data are available at \url{https://github.com/antonybevan/h-perforatum-network-tox}.

The manuscript is not under consideration elsewhere.

\bigskip



\end{document}

