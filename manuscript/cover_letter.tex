\documentclass[11pt]{article}
\usepackage[margin=1in]{geometry}
\usepackage{charter}
\usepackage{hyperref}
\usepackage{setspace}

\pagestyle{empty}

\begin{document}

\noindent
\textbf{Antony Bevan} \\
Department of Pharmacognosy \\
Madurai Medical College \\
Madurai, Tamil Nadu, India \\
Email: \href{mailto:antonybevan04@gmail.com}{antonybevan04@gmail.com}

\bigskip

\noindent
January 23, 2026

\bigskip

\noindent
Editor-in-Chief, \\
\textit{Scientific Reports} \\
Nature Portfolio

\bigskip

\noindent
Dear Editor,

\onehalfspacing

We are pleased to submit our manuscript, \textbf{``Perturbation efficiency resolves target-count bias in network proximity metrics: A controlled audit,''} for consideration as a Research Article in \textit{Scientific Reports}.

Our study identifies a fundamental statistical artifact in network-based drug prioritization. While proximity-based Z-scores are widely used to assess the association between compounds and diseases, we demonstrate through a systematic audit that these metrics are confounded by the Law of Large Numbers. This bias leads to deterministic significance inflation for compounds with broad polypharmacology, independent of actual topological reachability.

We address this by utilizing the human liver interactome and constituents from \textit{Hypericum perforatum} (St. John’s Wort) as a controlled model system. This system provides a sharp stress test for network medicine: the constituents exhibit extreme target-set asymmetry (10 vs.\ 62 targets), and the established proximity paradigm yields the biologically incorrect ranking. Our results show that:
\begin{itemize}
    \item Proximity-based prioritization is unstable across network construction parameters, yielding reversed rankings in asymmetric regimes.
    \item Random walk--based influence propagation provides a stable framework that capturing signal amplification through regulatory hubs like the PXR axis.
    \item Applying \textit{perturbation efficiency}—a per-target normalization of influence—resolves the size-dependence and correctly identifies high-leverage perturbations that distance metrics miss.
\end{itemize}

We believe this work is well-suited for \textit{Scientific Reports} as it provides a critical methodological resolution to a previously unrecognized artifact in network medicine. By identifying the boundary conditions for current proximity-based standards, we provide a reproducible template for correcting bias in complex pharmacological datasets—a finding with significant implications for the thousands of drug-disease associations previously computed using uncorrected Z-scores.

In line with the journal's commitment to Open Science, the complete end-to-end analytical pipeline, curated datasets, and reproducibility scripts are available at: \\ \url{https://github.com/antonybevan/h-perforatum-network-tox}.

This manuscript has not been published and is not under consideration for publication elsewhere. All authors have approved the final version and agree to the submission.

\bigskip

\noindent
Sincerely,

\vspace{1em}

\noindent
\textbf{Antony Bevan} \\
Corresponding Author

\end{document}
