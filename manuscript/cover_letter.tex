\documentclass[11pt]{article}
\usepackage[margin=1in]{geometry}
\usepackage{charter}
\usepackage{hyperref}
\usepackage{setspace}

\pagestyle{empty}

\begin{document}

\noindent
\textbf{Antony Bevan} \\
Department of Pharmacognosy \\
Madurai Medical College \\
Madurai, Tamil Nadu, India \\
Email: \href{mailto:antonybevan04@gmail.com}{antonybevan04@gmail.com}

\bigskip

\noindent
January 23, 2026

\bigskip

\noindent
Editor-in-Chief, \\
\textit{Scientific Reports} \\
Nature Portfolio

\bigskip

\noindent
Dear Editor,

\onehalfspacing

We are pleased to submit our manuscript, \textbf{``Systematic bias in network proximity Z-scores: A comparative robustness audit using \textit{Hypericum perforatum} constituents,''} for consideration as a Research Article in \textit{Scientific Reports}.

Our study identifies a fundamental statistical artifact in network-based drug prioritization. While proximity-based Z-scores are widely used to assess the association between compounds and diseases, we demonstrate that these metrics are systematically biased by target set size (the Law of Large Numbers). This confounding effect can lead to misleading biological rankings, particularly when comparing compounds with asymmetric polypharmacology—a scenario common in the study of natural products and drug repurposing.

Using the human liver interactome and constituents from \textit{Hypericum perforatum} (St. John’s Wort) as a model system, we show that:
\begin{itemize}
    \item Conventional proximity Z-scores fail a basic robustness audit, reversing rankings based on arbitrary network construction thresholds.
    \item Random walk--based influence propagation provides a stable, theoretically consistent framework that correctly identifies high-leverage perturbations that distance-based metrics miss.
    \item Our newly introduced metric, \textit{perturbation efficiency}, enables unbiased comparison across compounds with vastly different target set sizes.
\end{itemize}

We believe this work is well-suited for the multidisciplinary readership of \textit{Scientific Reports}, as it provides both a critical methodological warning for network medicine and a reproducible template for identifying and correcting statistical artifacts in complex biological datasets. 

In line with the journal's commitment to Open Science, the complete end-to-end analytical pipeline, curated datasets, and reproducibility scripts are available at: \\ \url{https://github.com/antonybevan/h-perforatum-network-tox}.

This manuscript has not been published and is not under consideration for publication elsewhere. All authors have approved the final version and agree to the submission.

\bigskip

\noindent
Sincerely,

\vspace{1em}

\noindent
\textbf{Antony Bevan} \\
Corresponding Author

\end{document}
