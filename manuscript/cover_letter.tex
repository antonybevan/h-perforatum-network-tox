% Cover Letter for Computational Toxicology
% Prepared for manuscript submission

\documentclass[11pt]{letter}
\usepackage[margin=1in]{geometry}
\usepackage{hyperref}
\usepackage{xcolor}

\signature{Antony Bevan\\
Department of Pharmacognosy\\
Madurai Medical College\\
Madurai, India\\
\texttt{antonybevan04@gmail.com}}

\address{Antony Bevan\\
Department of Pharmacognosy\\
Madurai Medical College\\
Madurai 625020, India}

\begin{document}

\begin{letter}{Editor-in-Chief\\
\textit{Computational Toxicology}\\
Elsevier}

\opening{Dear Editor,}

We are pleased to submit our manuscript entitled ``\textbf{Comparative analysis of network-based measures for the assessment of drug-induced liver injury: A case study of \textit{Hypericum perforatum}}'' for consideration as an Original Research Article in \textit{Computational Toxicology}.

\textbf{Summary.} This work addresses a critical methodological problem in network-based toxicology: the instability of proximity Z-scores when comparing compounds with asymmetric target set sizes. Using \textit{Hypericum perforatum} as a controlled case study, we demonstrate that proximity Z-scores can yield misleading rankings due to sample-size-dependent null distribution tightening (the law of large numbers). We show that random walk--based influence propagation provides a more robust framework, and introduce per-target network influence (PTNI) as an effect-size metric for comparative toxicological assessment.

\textbf{Key Findings.}
\begin{itemize}
    \item Proximity Z-scores are threshold-dependent and reverse compound rankings between STRING confidence levels.
    \item Influence-based metrics (RWR, EWI) remain stable across network construction parameters.
    \item Hyperforin targets achieve 3.7-fold greater DILI-directed perturbation efficiency than Quercetin targets, consistent with clinical hepatotoxicity profiles.
    \item Bootstrap sensitivity analysis excludes target-count confounding as an explanation.
\end{itemize}

\textbf{Relevance to Computational Toxicology.} Our work directly addresses the journal's focus on computational approaches for toxicological evaluation by:
\begin{enumerate}
    \item Identifying a systematic artifact in widely-used network proximity metrics.
    \item Providing a validated alternative framework (influence propagation + PTNI).
    \item Demonstrating practical robustness across multiple validation approaches.
    \item Contributing reproducible code and data for the community.
\end{enumerate}

\textbf{Reproducibility.} All code and data are publicly available at: \url{https://github.com/antonybevan/h-perforatum-network-tox}. Random seeds are fixed (seed=42) throughout, and all software versions are documented.

\textbf{Declarations.} This manuscript has not been published and is not under consideration elsewhere. The author declares no competing interests. No external funding was received. AI-assisted tools were used to support code development; the author takes full responsibility for all content.

We believe this work will be of significant interest to the readers of \textit{Computational Toxicology} as it provides both a methodological advance and a reproducible template for network-based risk assessment.

Thank you for considering our submission. We look forward to your response.

\closing{Sincerely,}

\end{letter}
\end{document}
